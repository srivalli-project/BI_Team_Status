In this challenge, you will use logical bitwise operators. All data is stored in its binary representation. The logical operators, and C language, use  to represent true and  to represent false. The logical operators compare bits in two numbers and return true or false,  or , for each bit compared.

Bitwise AND operator & The output of bitwise AND is 1 if the corresponding bits of two operands is 1. If either bit of an operand is 0, the result of corresponding bit is evaluated to 0. It is denoted by &.

Bitwise OR operator | The output of bitwise OR is 1 if at least one corresponding bit of two operands is 1. It is denoted by |.

Bitwise XOR (exclusive OR) operator ^ The result of bitwise XOR operator is 1 if the corresponding bits of two operands are opposite. It is denoted by .

For example, for integers 3 and 5,

3 = 00000011 (In Binary)
5 = 00000101 (In Binary)

AND operation        OR operation        XOR operation
  00000011             00000011            00000011
& 00000101           | 00000101          ^ 00000101
  ________             ________            ________
  00000001  = 1        00000111  = 7       00000110  = 6
You will be given an integer , and a threshold, i1nnik$. Print the results of the and, or and exclusive or comparisons on separate lines, in that order.

Example


The results of the comparisons are below:

a b   and or xor
1 2   0   3  3
1 3   1   3  2
2 3   2   3  1
For the and comparison, the maximum is . For the or comparison, none of the values is less than , so the maximum is . For the xor comparison, the maximum value less than  is . The function should print:

2
0
2
Function Description

Complete the calculate_the_maximum function in the editor below.

calculate_the_maximum has the following parameters:

int n: the highest number to consider
int k: the result of a comparison must be lower than this number to be considered
Prints

Print the maximum values for the and, or and xor comparisons, each on a separate line.

Input Format

The only line contains  space-separated integers,  and .

Constraints

Sample Input 0

5 4
Sample Output 0

2
3
3
Explanation 0



All possible values of  and  are:


The maximum possible value of  that is also  is , so we print  on first line.

The maximum possible value of  that is also  is , so we print  on second line.

The maximum possible value of  that is also  is , so we print  on third line.



void calculate_the_maximum(int n, int k) {
  //Write your code here.
  int a,b,am=0,om=0,xm=0;
  if(k<=n)
  {
      for(a=1;a<=n;a++)
      {
          for(b=a+1;b<=n;b++)
          {
            if(((a&b)<k) && (am<(a&b)))
                am=a&b;
            if(((a|b)<k) && (om<(a|b)))
                om=a|b;
            if(((a^b)<k) && (xm<(a^b)))
                xm=a^b;
          }
      }
  }
  printf("%d\n%d\n%d",am,om,xm);
}
